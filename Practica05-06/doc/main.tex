\documentclass{rapportECL}
\usepackage{amsmath}
\usepackage{fancybox}
\usepackage{lipsum}
\usepackage{biblatex}
\usepackage{csquotes}
\addbibresource{main.bib}
\usepackage{times}
\usepackage{fancyhdr,graphicx,amsmath,amssymb}
\usepackage[spanish]{babel}
\usepackage{listings}
\usepackage{xcolor}
\definecolor{codegreen}{rgb}{0,0.6,0}
\definecolor{codegray}{rgb}{0.5,0.5,0.5}
\definecolor{codepurple}{rgb}{0.58,0,0.82}
\definecolor{backcolour}{rgb}{0.95,0.95,0.92}

\lstdefinestyle{mystyle}{
    backgroundcolor=\color{backcolour},   % backcolour
    commentstyle=\color{codegreen},
    keywordstyle=\color{blue},
    numberstyle=\tiny\color{codegray},
    stringstyle=\color{codegreen},
    basicstyle=\ttfamily\footnotesize,
    breakatwhitespace=false,         
    breaklines=true,                 
    captionpos=b,                    
    keepspaces=true,                 
    numbers=left,                    
    numbersep=5pt,                  
    showspaces=false,                
    showstringspaces=false,
    showtabs=false,                  
    tabsize=2
}
\lstset{style=mystyle}\usepackage{xcolor}
\DeclareUnicodeCharacter{2212}{-}
\DeclareUnicodeCharacter{2217}{*}
\newcommand{\subsubsubsection}[1]{\paragraph{#1}\mbox{}\\}
\setcounter{secnumdepth}{4}
\setcounter{tocdepth}{4}

\title{Practica05-06_EDA}

\begin{document}


\titre{Pr\'actica 05 y 06}
\UE{\textbf{Escuela Profesional de Ciencia de la Computaci\'on}} %Nom de la UE
\sujet{Estructuras de Datos Avanzados} %Nom du sujet

\eleves{- Castillo Caccire, Kemely Francis\\
        - Chullunquía Rosas, Sharon Rossely\\
        - Santos Apaza, Yordy Williams\\
        - Zúñiga Coayla, Jerson}
\enseignant{Mg. Vicente Machaca Arceda}

%----------- Initialisation -------------------
        
\fairemarges %Afficher les marges
\fairepagedegarde %Créer la page de garde 
\tableofcontents
\newpage

\section{Introducción}

El  particionado  del  espacio  por  estructuras  arbóreas  divide  una  región  en  varias  partes, que pueden ser divididas en otras partes más pequeñas sucesivamente. Para entender los kd-trees es necesario comprender primero cómo funcionan los quadtrees/octrees.Quadtrees y octrees son semejantes, salvo que cuando se trabaja en dos dimensiones se usan quadtrees (cada nodo despliega cuatro hijos) y en tres dimensiones octrees(cada nodo despliega ocho hijos). \\

Un  kd-tree  es un  árbol  binario  en  el  que  cada  división  de  una  región se  parte  en dos regiones disjuntas. Cada una de ellas puede dividirse en dos más pequeñas sucesivamente hasta los  nodos  hoja,donde se  guarda  el  listado  de  los  objetos  que intersecan  con su  región.A diferencia del quadtree la división no tiene por qué generar dos particiones iguales. Para hacer el corte  o  división,  se  utiliza una  recta  (para  dos  dimensiones)  o  un  plano  (para  tres  dimensiones) alineados con los ejes de coordenadas.

\begin{figure}[H]
  \centering
  \includegraphics[width=0.8\textwidth]{images/KDtree.png}
  \caption{(a)La descomposición del árbol kd para una región. (b)El árbol kd para la región a}
  \label{fig:act-kdtree}
\end{figure}
\section{Ejercicios}
\subsection{main.html}
Cree un archivo main.html:
Creamos un archivo main.html , el cual llamara a los archivos  javascript , el archivo p5.min.js es una librería para gráfícos.En el archivo kdtree.js se encontrara la estructura ,mientras que en sketch.js prepara el ambiente de trabajo donde haremos las pruebas .
\begin{lstlisting}[language=c++,
                   directivestyle={\color{black}}
                   emph={int,char,double,float,unsigned},
                   emphstyle={\color{blue}}
                  ]
<html >
<head >
<title >Kd tree </ title >
<script src ="p5.min.js" ></ script >
<script src =" kdtree.js" ></ script >
<script src =" sketch.js" ></ script >
</head >
<body >
</body >
</html >
\end{lstlisting}

\begin{figure}[H]
  \centering
  \includegraphics[width=1\textwidth]{images/uno.PNG}
  \caption{main.html}
  \label{fig:act-1}
\end{figure}


\subsection{Range query rectangle}
Implemente la función range\_query\_rec del KD-Tree, esta vez el range representa un rectángulo.
\begin{lstlisting}[language=C++,
                   directivestyle={\color{black}}
                   emph={int,char,double,float,unsigned},
                   emphstyle={\color{blue}}
                  ]
function rangeQueryRect(node, puntoConsulta, kpoints, rectangle, depth = 0) {
  if (!node) return null;

  var subTree1 = node.left;
  var subTree2 = node.right;

  if (puntoConsulta[depth%k] >= node.point[depth%k]) {
    subTree1 = node.right;
    subTree2 = node.left;
  }

  masCercano(puntoConsulta, rangeQueryRect(subTree1, puntoConsulta, kpoints, rectangle, depth + 1), node.point);
  if (node.point[0] > (rectangle.x - rectangle.w) &&
      node.point[0] < (rectangle.x + rectangle.w) &&
      node.point[1] > (rectangle.y - rectangle.h) &&
      node.point[1] < (rectangle.y + rectangle.h)) {
    kpoints.push(node.point)
  }

  let distanceSector = Math.abs(puntoConsulta[depth%k] - node.point[depth%k]);
  if (distanceSector <= rectangle.x + rectangle.w ||
      distanceSector <= rectangle.x - rectangle.w ||
      distanceSector <= rectangle.y + rectangle.h ||
      distanceSector <= rectangle.y - rectangle.h) {
        masCercano(puntoConsulta, rangeQueryRect(subTree2, puntoConsulta, kpoints, rectangle, depth + 1), node.point);
  }
}
\end{lstlisting}
\begin{figure}[H]
  \centering
  \includegraphics[width=0.7\textwidth]{images/7a.PNG}
  \label{fig:act-7a}
\end{figure}

\subsection{Método \textit{Reduction}}
\doublebox{
    \begin{minipage}[c][1.2\height] [c]{1\textwidth}
    \newline
    Defina un método \textit{reduction}, este método eliminará el último nivel del Octree y aculumará los valores de los canales RGB al nodo padre.
    \end{minipage}
}\\
\begin{lstlisting}[language=C++,
                   directivestyle={\color{black}}
                   emph={int,char,double,float,unsigned},
                   emphstyle={\color{blue}}
                  ]
void OctreeQuantizer::reduction() {
  if (!levels) {
    return;
  }

  root->eliminar();
  --levels;
}
void Node::eliminar() {
  if (hijo[0]->hoja) {
    for (int i = 0; i < 8; i++) {
      pixel += hijo[i]->pixel;
      color.b += hijo[i]->color.b;
      color.g += hijo[i]->color.g;
      color.r += hijo[i]->color.r;
      delete hijo[i];
    }
    hoja = true;
    return;
  }

  for (int i = 0; i < 8; i++) {
    hijo[i]->eliminar();
  }
}
\end{lstlisting}
\subsection{Implemente la función closest\_point\_brute\_force y naive\_closest\_point}

\begin{lstlisting}[language=C++,
                   directivestyle={\color{black}}
                   emph={int,char,double,float,unsigned},
                   emphstyle={\color{blue}}
                  ]
k=2;
class Node {
	constructor (point,axis){
		this.point = point;
		this.left = null;
		this.right = null;
        this.axis = axis;
	}
}
function distanceSquared(pointA, pointB) {
  var distance = 0;
  for (var i = 0; i < k; i++) {
    distance += Math.pow((pointA[i]-pointB[i]), 2)
  }
  return Math.sqrt(distance);
}
function closest_point_brute_force ( points , point ){}
function naive_closest_point (node , point , depth = 0, best = null ){}
\end{lstlisting}

\subsubsection{closest\_point\_brute\_force}
\begin{lstlisting}[language=C++,
                   directivestyle={\color{black}}
                   emph={int,char,double,float,unsigned},
                   emphstyle={\color{blue}}
                  ]
function closest_point_brute_force(points, point){
  var distance = Number.MAX_VALUE;
  var point_c = null;
  for (var i = 0; i< points.length; i++){
      var auxiliar = distanceSquared(points[i], point);
      if (auxiliar < distance){
          distance = auxiliar;
          point_c = points[i];
      }
  }
  return distance;
}
\end{lstlisting}

\subsubsection{naive\_closest\_point}
\begin{lstlisting}[language=C++,
                   directivestyle={\color{black}}
                   emph={int,char,double,float,unsigned},
                   emphstyle={\color{blue}}
                  ]
function naive_closest_point (node , point , depth = 0, best = null ){
  if (!node) {
    return best;
  }
  var ladoIZQ = node.left;
  var ladoDER = node.right;
  if(best == null || distanceSquared(node.point ,point) < distanceSquared(best,point)){
    best = node.point;
  }
  if (point[depth % k] < node.point[depth % k]) {
    return naive_closest_point(ladoIZQ, point, depth +1 , best)
  }
  else{
    return naive_closest_point(ladoDER, point, depth +1 , best)  
  }
}
\end{lstlisting}
\subsection{Paleta generada}
\doublebox{
    \begin{minipage}[c][1.2\height] [c]{1\textwidth}
    \newline
    Defina un método \textit{pallete}, este construirá la paleta.
    \end{minipage}
}\\

\begin{lstlisting}[language=C++,
                   directivestyle={\color{black}}
                   emph={int,char,double,float,unsigned},
                   emphstyle={\color{blue}}
                  ]
void OctreeQuantizer::palette(cv::Mat &entry) {
  std::vector<Color> colors;
  push_colors(root, colors);

  int canal = entry.channels();
  int filas = entry.rows;
  int columnas = entry.cols * canal;

  uchar *p;

  uint step_size = filas / colors.size();
  uint step = step_size;
  int c_i = 0;
  for (int i = 0; i < filas; ++i) {
    p = entry.ptr<uchar>(i);
    for (int j = 0; j < columnas; j += 3) {
      p[j] = colors[c_i].b;
      p[j + 1] = colors[c_i].g;
      p[j + 2] = colors[c_i].r;
    }
    if (i > step) {
      ++c_i;
      step += step_size;
    }
  }
}

void OctreeQuantizer::push_colors(Node *root, std::vector<Color> &colors) {
  if (root == nullptr) {
    return;
  }
  if (root->hoja && root->pixel) {
    colors.push_back(Color(root->color.r / root->pixel,
                           root->color.g / root->pixel,
                           root->color.b / root->pixel));
    return;
  }
  for (uint i = 0; i < 8; i++) {
    push_colors(root->hijo[i], colors);
  }
}
\end{lstlisting}
Como cada hoja tiene el número de píxeles con el color y la suma del color de los valores R, G y B, el color promedio se puede recibir dividiendo los canales de color por el número de píxeles: colors.push\_back(Color(root->color.r / pixel\_count, root->color.g / pixel\_count, root->color.b / pixel\_count));
\subsection{Prueba \#2 :}
Evalue el resultado de las dos funciones implementadas anteriormente con este conjunto de datos:\\

\doublebox{
    \begin{minipage}[c][1.2\height] [c]{1\textwidth}
var data = [\newline
[40 ,70] ,\newline
[70 ,130] ,\newline
[90 ,40] ,\newline
[110 , 100] ,\newline
[140 ,110] ,\newline
[160 , 100] ,\newline
[150 , 30]\newline
];\newline
var point = [140 ,90]; // query
    \end{minipage}
}
\begin{figure}[H]
 \centering
 \includegraphics[width=0.9\textwidth]{images/prueba2.PNG}
 \label{fig:act-5-prueba2}
\end{figure}

\begin{itemize}
    \item closest\_point\_brute\_force
    \begin{figure}[H]
     \centering
     \includegraphics[width=0.5\textwidth]{images/prueba2_brute_force.PNG}
     \label{fig:act-6-1}
     \caption{Podemos apreciar el punto mas cercano de color \textcolor{green}{verde} el cual es [140,110].}
    \end{figure}
    \item naive\_closest\_point
    \begin{figure}[H]
     \centering
     \includegraphics[width=0.6\textwidth]{images/prueba2_naive.PNG}
     \label{fig:act-6-2}
     \caption{Podemos apreciar el punto mas cercano de color \textcolor{green}{verde} el cual es [160,100].}
    \end{figure}
\end{itemize}

Observamos que en la función closest\_point\_brute\_force el punto mas cercano es [140,110] y en naive\_closest\_point el punto mas cercano es [160,100].




\newpage
\begin{thebibliography}{8}

\bibitem{concept}
OpenCv Introduction \textit{What is OpenCv?}. \url{https://docs.opencv.org/master/d1/dfb/intro.html}.

\bibitem{concept2}
Applications: Octree Color Quantization. \url{https://en.wikipedia.org/wiki/Color_quantization}.

\end{thebibliography}

%\printbibliography[heading=bibintoc]

\end{document}