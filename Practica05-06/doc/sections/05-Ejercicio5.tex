\subsection{Paleta generada}
\doublebox{
    \begin{minipage}[c][1.2\height] [c]{1\textwidth}
    \newline
    Defina un método \textit{pallete}, este construirá la paleta.
    \end{minipage}
}\\

\begin{lstlisting}[language=C++,
                   directivestyle={\color{black}}
                   emph={int,char,double,float,unsigned},
                   emphstyle={\color{blue}}
                  ]
void OctreeQuantizer::palette(cv::Mat &entry) {
  std::vector<Color> colors;
  push_colors(root, colors);

  int canal = entry.channels();
  int filas = entry.rows;
  int columnas = entry.cols * canal;

  uchar *p;

  uint step_size = filas / colors.size();
  uint step = step_size;
  int c_i = 0;
  for (int i = 0; i < filas; ++i) {
    p = entry.ptr<uchar>(i);
    for (int j = 0; j < columnas; j += 3) {
      p[j] = colors[c_i].b;
      p[j + 1] = colors[c_i].g;
      p[j + 2] = colors[c_i].r;
    }
    if (i > step) {
      ++c_i;
      step += step_size;
    }
  }
}

void OctreeQuantizer::push_colors(Node *root, std::vector<Color> &colors) {
  if (root == nullptr) {
    return;
  }
  if (root->hoja && root->pixel) {
    colors.push_back(Color(root->color.r / root->pixel,
                           root->color.g / root->pixel,
                           root->color.b / root->pixel));
    return;
  }
  for (uint i = 0; i < 8; i++) {
    push_colors(root->hijo[i], colors);
  }
}
\end{lstlisting}
Como cada hoja tiene el número de píxeles con el color y la suma del color de los valores R, G y B, el color promedio se puede recibir dividiendo los canales de color por el número de píxeles: colors.push\_back(Color(root->color.r / pixel\_count, root->color.g / pixel\_count, root->color.b / pixel\_count));