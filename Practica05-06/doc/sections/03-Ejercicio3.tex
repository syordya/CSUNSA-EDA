\subsection{Método \textit{Reduction}}
\doublebox{
    \begin{minipage}[c][1.2\height] [c]{1\textwidth}
    \newline
    Defina un método \textit{reduction}, este método eliminará el último nivel del Octree y aculumará los valores de los canales RGB al nodo padre.
    \end{minipage}
}\\
\begin{lstlisting}[language=C++,
                   directivestyle={\color{black}}
                   emph={int,char,double,float,unsigned},
                   emphstyle={\color{blue}}
                  ]
void OctreeQuantizer::reduction() {
  if (!levels) {
    return;
  }

  root->eliminar();
  --levels;
}
void Node::eliminar() {
  if (hijo[0]->hoja) {
    for (int i = 0; i < 8; i++) {
      pixel += hijo[i]->pixel;
      color.b += hijo[i]->color.b;
      color.g += hijo[i]->color.g;
      color.r += hijo[i]->color.r;
      delete hijo[i];
    }
    hoja = true;
    return;
  }

  for (int i = 0; i < 8; i++) {
    hijo[i]->eliminar();
  }
}
\end{lstlisting}