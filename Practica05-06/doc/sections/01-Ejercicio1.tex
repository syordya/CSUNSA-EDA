\section{Ejercicios}
\subsection{Estructuras \textit{Color, Node, Octree}}
\doublebox{
    \begin{minipage}[c][1.2\height] [c]{1\textwidth}
    \newline
    Defina la clase \textit{node} del Octree. El nodo debe almacenar un color en RGB, un contador (para saber cuantos pixeles tienen ese color) y un \textit{flag} para saber si es nodo hoja.
    \end{minipage}
}\\
\begin{lstlisting}[language=C++,
                   directivestyle={\color{black}}
                   emph={int,char,double,float,unsigned},
                   emphstyle={\color{blue}}
                  ]
class Color {
public:
    uint r, g, b;
    Color(uchar _r = 0, uchar _g = 0, uchar _b = 0);
};


class Node {
public:
    Color color;
    int pixel;
    bool hoja;
    int level;
    Node* hijo[8];
    Node(Color _color = Color(), int _pixel = 0, bool _hoja = false,
        int level = 0);
    ~Node();
    void agregar(int);
    void eliminar();
};

class OctreeQuantizer {
 private:
  int levels;
  Node *root;
  void push_colors(Node *root, std::vector<Color> &colors);

 public:
  OctreeQuantizer();
  ~OctreeQuantizer();
  void fill(cv::Mat &entry);
  void reduction();
  void reconstruction(cv::Mat &entry);
  void palette(cv::Mat &entry);
};
\end{lstlisting}


\iffalse

% ---- Para colocar una imagen ----
Como se muestra en la figura \ref{fig:act-3}
\begin{figure}[H]
  \centering
  \includegraphics[width=0.8\textwidth]{act-3}
  \caption{Tabla de subneteo para la red 192.168.100.0.}
  \label{fig:act-3}
\end{figure}

\fi