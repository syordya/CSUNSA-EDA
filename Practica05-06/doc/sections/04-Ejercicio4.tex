\subsection{Método \textit{Reconstruction}}
\doublebox{
    \begin{minipage}[c][1.2\height] [c]{1\textwidth}
    \newline
    Defina un método \textit{reconstruction}, que asignará a cada pixel su nuevo color, despues de aplicar la reducción.
    \end{minipage}
}\\
\begin{lstlisting}[language=C++,
                   directivestyle={\color{black}}
                   emph={int,char,double,float,unsigned},
                   emphstyle={\color{blue}}
                  ]
void OctreeQuantizer::reconstruction(cv::Mat &entry) {
  int canal = entry.channels();
  int filas = entry.rows;
  int columnas = entry.cols * canal;

  if (entry.isContinuous()) {
    columnas *= filas;
    filas = 1;
  }

  uchar *p;
  Node *ruta;
  int i = 0;
  while (i < filas) {
    p = entry.ptr<uchar>(i);
    int j = 0;
    while (j < columnas) {
      uchar b = p[j], g = p[j + 1], r = p[j + 2];
      ruta = root;
      for (int level = 0; level < levels; level++) {
        ruta = ruta->hijo[get_index_level(r, g, b, level)];
      }

      p[j] = (ruta->color.b) / (ruta->pixel);
      p[j + 1] = (ruta->color.g) / (ruta->pixel);
      p[j + 2] = (ruta->color.r) / (ruta->pixel);
      j += 3;
    }
    ++i;
  }
}
\end{lstlisting}