\subsection{Método \textit{Fill}}
\doublebox{
    \begin{minipage}[c][1.2\height] [c]{1\textwidth}
    \newline
    Defina un método \textit{fill} que lea los pixeles de una imagen y construya el Octree según el algoritmo explicado en clases.
    \end{minipage}
}\\
\begin{lstlisting}[language=C++,
                   directivestyle={\color{black}}
                   emph={int,char,double,float,unsigned},
                   emphstyle={\color{blue}}
                  ]
OctreeQuantizer::~OctreeQuantizer() { delete root; }

void OctreeQuantizer::fill(cv::Mat &entry) {
    int canal = entry.channels();
    int filas = entry.rows;
    int columnas = entry.cols * canal;

  if (entry.isContinuous()) {
    columnas *= filas;
    filas = 1;
  }

  uchar *p;
  Node *ruta;
  int i = 0;
  while (i < filas) {
    p = entry.ptr<uchar>(i);
    int j = 0;
    while (j < columnas) {
      uchar b = p[j], g = p[j + 1], r = p[j + 2];
      ruta = root;
      for (int level = 0; level < levels; level++) {
           ruta = ruta->hijo[get_index_level(p[j + 2], p[j + 1], p[j], level)];
      }

      ruta->color.b += p[j];
      ruta->color.g += p[j + 1];
      ruta->color.r += p[j + 2];

      ++(ruta->pixel);
      j += 3;
    }
    ++i;
  }
}
\end{lstlisting}