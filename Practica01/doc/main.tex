\documentclass[a4paper,12pt]{article}
\usepackage{custom}
\usepackage{import}

\title{Algoritmos de Ordenamiento}
\author{- Castillo Caccire, Kemely Francis\\
        - Chullunquía Rosas, Sharon Rossely\\
        - Santos Apaza, Yordy Williams\\
        - Zúñiga Coayla, Jerson (tu amorcito xdd)}
\date{\today}

\makeatletter
\let\thetitle\@title
\let\theauthor\@author
\let\thedate\@date
\makeatother
\pagestyle{fancy}
\fancyhf{}
\lhead{Estructuras de Datos Avanzados}
\rhead{\thetitle}
\cfoot{\thepage}



\begin{document}

\import{./}{title.tex}


\import{sections/}{01_pregunta01.tex}
\import{sections/}{02_pregunta02.tex}
\import{sections/}{03_pregunta03.tex}

% \printbibliography[heading=bibintoc]

\newpage
\begin{thebibliography}{8}

\bibitem{ref_url1}
Jacinto Ruiz Catalán, \textit{COMPILADORES. Teoría e implementación}, 2010.
\url{https://books.google.com.pe/books?id=yG6qJBAnE9UC&pg=PA1&dq=que+es+un+compilador+en+programacion&hl=es&sa=X&ved=2ahUKEwiHuc69s_XqAhXjkOAKHSL1BAEQ6AEwAnoECAAQAg#v=onepage&q&f=false} 

\bibitem{ref_url2}
Alfred Aho, Ravi Sethi, Jeffrey Ullman, Monica S. Lam. \textit{Compilers: Principles, Techniques, and Tools}.

\bibitem{ref_url3}
Stephen C. Johnson. YACC: Yet another compiler-compiler. \textit{Unix Programmer's Manual Vol 2b}.


\end{thebibliography}

\end{document}
