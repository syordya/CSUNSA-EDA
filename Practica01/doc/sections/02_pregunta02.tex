\section*{Ejercicio \#2}
En un paso siguiente procedimos a implementar los algoritmos de ordenamiento:
\begin{itemize}
    \item \href{https://es.wikipedia.org/wiki/Ordenamiento_de_burbuja} {\textbf{Bubble Sort} } : Complejidad O(n$^$2)
    \item \href{https://es.wikipedia.org/wiki/Ordenamiento_por_cuentas} {\textbf{Counting sort} } : Complejidad O(n+k)
    \item \href{https://es.wikipedia.org/wiki/Heapsort} {\textbf{Heap sort} } : Complejidad O(n log n )
    \item \href{https://es.wikipedia.org/wiki/Ordenamiento_por_inserción } {\textbf{Insertion sort} } : Complejidad O(n$^$2)
    \item \href{https://es.wikipedia.org/wiki/Ordenamiento_por_mezcla} {\textbf{Merge sort} } : Complejidad O(n log n)
    \item \href{hhttps://es.wikipedia.org/wiki/Quicksort} {\textbf{Quick sort} } : Complejidad O(n log n)
    \item \href{https://es.wikipedia.org/wiki/Ordenamiento_por_selección} {\textbf{Selection sort} } Complejidad O(n$^$2)
\end{itemize}
Para esta tarea usamos los lenguajes C++, Java y Python y los códigos obtenidos los pueden encontrar en \url{https://github.com/syordya/CSUNSA-EDA/tree/master/Practica01/code}.


% ---- Para poner dos imágenes (una a lado de otra) ----
\iffalse
Como se muestra en la figuras \ref{fig:act-1_a} y \ref{fig:act-1_b}.
\begin{figure}[H]
\centering
\begin{minipage}{0.45\textwidth}
  \centering
  \includegraphics[width=0.9\textwidth]{act-1_a}
  \caption{Envío de \textit{ICMP ECHO REQUEST} de PC0 a PC1, PC2 y PC3.}
  \label{fig:act-1_a}
\end{minipage}\hfill
\begin{minipage}{0.45\textwidth}
  \centering
  \includegraphics[width=0.9\textwidth]{act-1_b}
  \caption{Respuesta de PC1, PC2 y PC3. Tabla ARP de PC0.}
  \label{fig:act-1_b}
\end{minipage}
\end{figure}
% ---- Para colocar una imagen ----
Como se muestra en la figura \ref{fig:act-3}
\begin{figure}[H]
  \centering
  \includegraphics[width=0.8\textwidth]{act-3}
  \caption{Tabla de subneteo para la red 192.168.100.0.}
  \label{fig:act-3}
\end{figure}
\fi