\section*{Ejercicio \#3}
Primeramente, graficaremos el tiempo de procesamiento de cada algoritmo en los tres lenguajes anteriormente mencionados : Java, C++ y Python. Seguidamente y por último se presentará los gráficos comparativos de todos los algoritmos por cada lenguaje, donde compararemos los tiempos de procesamiento y se hará un breve análisis acompañado de cuadros comparativos que nos ayudarán a dar una mejor examinación.
% Hola \cite{cormen}.

\begin{itemize}
    \item \textbf{Gráficas comparativas de cada algoritmo en Java, C++ y Python}
    
    Comparar el tiempo de procesamiento de los tres lenguajes de programaci ́on por cada algo-ritmo. 
    \item 
\end{itemize}

% ---- Para poner dos imágenes (una a lado de otra) ----


























\iffalse
Como se muestra en la figuras \ref{fig:act-1_a} y \ref{fig:act-1_b}.
\begin{figure}[H]
\centering
\begin{minipage}{0.45\textwidth}
  \centering
  \includegraphics[width=0.9\textwidth]{act-1_a}
  \caption{Envío de \textit{ICMP ECHO REQUEST} de PC0 a PC1, PC2 y PC3.}
  \label{fig:act-1_a}
\end{minipage}\hfill
\begin{minipage}{0.45\textwidth}
  \centering
  \includegraphics[width=0.9\textwidth]{act-1_b}
  \caption{Respuesta de PC1, PC2 y PC3. Tabla ARP de PC0.}
  \label{fig:act-1_b}
\end{minipage}
\end{figure}
% ---- Para colocar una imagen ----
Como se muestra en la figura \ref{fig:act-3}
\begin{figure}[H]
  \centering
  \includegraphics[width=0.8\textwidth]{act-3}
  \caption{Tabla de subneteo para la red 192.168.100.0.}
  \label{fig:act-3}
\end{figure}
\fi