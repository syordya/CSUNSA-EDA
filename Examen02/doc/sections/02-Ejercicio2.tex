\subsection{kdtree.js}
Cree un archivo kdtree.js.
\begin{lstlisting}[language=C++,
                   directivestyle={\color{black}}
                   emph={int,char,double,float,unsigned},
                   emphstyle={\color{blue}}
                  ]
k = 2;
class Node {
constructor (point , axis ){
this . point = point ;
this . left = null ;
this . right = null ;
this . axis = axis ;
}
}
function getHeight ( node ) {}
function generate_dot ( node ){}
function build_kdtree ( points , depth = 0) {}
\end{lstlisting}

Complete las funciones:
\subsubsection{build\_kdtree }Construye el KD-Tree y retorna el nodo raiz.
\begin{lstlisting}[language=C++,
                   directivestyle={\color{black}}
                   emph={int,char,double,float,unsigned},
                   emphstyle={\color{blue}}
                  ]
function build_kdtree(points, depth = 0) {

  var median;
  var root;
  
  if (points.length == 0) {
    return null;
  }
  else {
    var axis = depth % k; // par 0 (x), impar 1 (y)
    
    points = points.sort(
      function (a, b) {
        if (a[axis] < b[axis]) return -1;
        if (a[axis] > b[axis]) return 1;
        return 0;
      });
    if (points.length == 1) {
      median = 0;
    }
    else {
      median = Math.floor(points.length / 2); 
    }

    root = new Node(points[median], axis);
    
    root.left = build_kdtree(points.slice(0, median), depth + 1);
    root.right = build_kdtree(points.slice(median + 1,), depth + 1);

    return root;
  }
}
\end{lstlisting}

\newpage
\subsubsection{getHeight }Retorna la altura del árbol.
\begin{lstlisting}[language=C++,
                   directivestyle={\color{black}}
                   emph={int,char,double,float,unsigned},
                   emphstyle={\color{blue}}
                  ]
function getHeight(node) {
  if (!node) {
    return -1;
  }
  if (node.left == null && node.right == null) {
    return 0;
  }
  var left_child;
  var right_child
  left_child = getHeight(node.left) + 1;
  right_child = getHeight(node.right) + 1;

  return Math.max(left_child, right_child);
}
\end{lstlisting}
\subsubsection{generate\_dot }Genera al árbol en formato dot, por ejemplo:
\begin{lstlisting}[language=C++,
                   directivestyle={\color{black}}
                   emph={int,char,double,float,unsigned},
                   emphstyle={\color{blue}}
                  ]
digraph G {
"106,189" -> "6,114";
"6,114" -> "90,102";
"90,102" -> "21,84";
"6,114" -> "84,138 ";
"84,138" -> "5,150";
"106,189" -> "148,85";
"148,85" -> "181,45";
"181,45" -> "161,29";
"148,85" -> "158,120";
}
\end{lstlisting}

\begin{lstlisting}[language=C++,
                   directivestyle={\color{black}}
                   emph={int,char,double,float,unsigned},
                   emphstyle={\color{blue}}
                  ]
function generateDot(node) {
  if (!node || getHeight(node) == 0) {
    return '';
  }

  let str = '';
  if (node.left) {
    str += '"' + node.point[0] + ',' + node.point[1] + '"';
    str += " -> ";
    str += '"' + node.left.point[0] + ',' + node.left.point[1] + '"';
    str += ';\n';
    str += generateDot(node.left);
  }
  if (node.right) {
    str += '"' + node.point[0] + ',' + node.point[1] + '"';
    str += " -> ";
    str += '"' + node.right.point[0] + ',' + node.right.point[1] + '"';
    str += ';\n';
    str += generateDot(node.right);
  }
  return str;
}
\end{lstlisting}
