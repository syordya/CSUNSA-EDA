\subsection{Range query rectangle}
Implemente la función range\_query\_rec del KD-Tree, esta vez el range representa un rectángulo.
\begin{lstlisting}[language=C++,
                   directivestyle={\color{black}}
                   emph={int,char,double,float,unsigned},
                   emphstyle={\color{blue}}
                  ]
function rangeQueryRect(node, puntoConsulta, kpoints, rectangle, depth = 0) {
  if (!node) return null;

  var subTree1 = node.left;
  var subTree2 = node.right;

  if (puntoConsulta[depth%k] >= node.point[depth%k]) {
    subTree1 = node.right;
    subTree2 = node.left;
  }

  masCercano(puntoConsulta, rangeQueryRect(subTree1, puntoConsulta, kpoints, rectangle, depth + 1), node.point);
  if (node.point[0] > (rectangle.x - rectangle.w) &&
      node.point[0] < (rectangle.x + rectangle.w) &&
      node.point[1] > (rectangle.y - rectangle.h) &&
      node.point[1] < (rectangle.y + rectangle.h)) {
    kpoints.push(node.point)
  }

  let distanceSector = Math.abs(puntoConsulta[depth%k] - node.point[depth%k]);
  if (distanceSector <= rectangle.x + rectangle.w ||
      distanceSector <= rectangle.x - rectangle.w ||
      distanceSector <= rectangle.y + rectangle.h ||
      distanceSector <= rectangle.y - rectangle.h) {
        masCercano(puntoConsulta, rangeQueryRect(subTree2, puntoConsulta, kpoints, rectangle, depth + 1), node.point);
  }
}
\end{lstlisting}
\begin{figure}[H]
  \centering
  \includegraphics[width=0.7\textwidth]{images/7a.PNG}
  \label{fig:act-7a}
\end{figure}
