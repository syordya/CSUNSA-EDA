\section{Introducci\'on}
\item Los empaquetadores, comúnmente llamados en términos anglosajones, bundlers, son ya una herramienta imprescindible en cualquier entorno web moderno que se precie. Con la reciente ganancia de popularidad de Javascript, gracias a Nodejs, estas herramientas han proliferado en forma de utilidad de consola. Nodejs permite crear con gran agilidad este tipo de herramientas, e incluso escribir los propios módulos.
Grunt y Gulp, fueron los que sentaron las bases de lo que son actualmente herramientas como webpack. Tenían por función la ejecución tareas de cualquier tipo, con plugins para ampliar la funcionalidad. Actualmente, están proliferando diferentes soluciones al problema mas específico de agrupar recursos, como puede ser rollup o jspm entre otros. Todos ellos están escritos en Nodejs.\\

\item Nodejs, de ahora en adelante, node, ha supuesto un cambio en la forma de desarrollo de la web por lo que el lado del servidor corresponde. De la propia de definición de su página oficial: usa un node ha supuesto un cambio en la forma de desarrollo de la web por lo que el lado del servidor corresponde. De la propia de definición de su página oficial: usa un  modelo de operaciones Entrada/Salida sin bloqueo y orientado a eventos, que lo hace liviano y eficiente y programación funcional. Cuenta con una comunidad muy activa que desarrolla todo tipo de funcionalidades en forma de paquetes que reside en npm. La equivalencia podría ser packagist, en este caso, respecto de paquetes de php.  La proliferación de páginas SPA, con el uso intensivo de red o RTA combinado ha propiciado la amplia expansión y evolución. Además de poder unificar la lógica en  cliente y servidor, y usar el mismo lenguaje de programación para ambos.