\section{Definici\'on}


\doublebox{
    \begin{minipage}[c][1.2\height] [c]{1\textwidth}
    \newline
    Es un empaquetador de archivos para aplicaciones JavaScript modernas, totalmente configurable y a diferencia de los Task Runners (como Grunt y Gulp) donde los procesos se gestionan de forma separada, en webpack, se conoce el origen y todo se compila en un único archivo.

    Crea una gráfica de todas las dependencias de la aplicación. Tiene un archivo de configuración, denominado webpack.config.js, donde se define todo el proceso de construcción en un objeto JS.
    \end{minipage}
}\\

\item Webpack tiene 4 conceptos clave:
\begin{enumerate}
    \item \textbf{Entry:} Indica cuál es el punto(s) de entrada.
    \item \textbf{Output:} Indica cuál es el punto(s) de salida.
    \item \textbf{Loaders:} Realizan transformaciones en los archivos.
    \item \textbf{Plugins:} Realizan acciones en los archivos.
\end{enumerate}

\section{Beneficios}
\begin{itemize}
    \item Es una herramienta que nos permite automatizar tareas repetitivas.
    \item Modularizar las diferentes secciones de nuestro proyecto.
    \item Posibilidad de elección en función del tipo de asset(minificación de imágenes, etc).
    \item Tiene una comunidad muy activa, y cuenta con un gran apoyo por parte de empresas como egghead.io, rollbar, etc.
    \item Aunque en su primera versión era uno de los principales inconvenientes, en su segunda versión cuenta con una documentación de calidad.
    \item Optimiza la velocidad de carga de la web.
\end{itemize}

\section{Desventajas}
\begin{itemize}
    \item En proyectos en los que la gestión de los assets no represente un problema, o el proyecto sea pequeño.
    \item La separación entre las secciones sea difusa o difícilmente modularizable.
    \item La curva de aprendizaje es elevada, necesita mucha configuración y es compleja al principio.
\end{itemize}